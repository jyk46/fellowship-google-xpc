%=========================================================================
% fig-xpc-api.tex
%=========================================================================

%\begin{figure*}

  \cbxsetfontsize{8pt}
  \begin{subfigure}{\tw}
  \begin{Verbatim}[gobble=6]
    void bfs( Node nodes[] ) {
      xpc::spawn( start, func = [&] (int idx) {

        Node my_node = node[idx];

        for ( i = 0; i < my_node.num_neigh; i++ ) {

          Node neighbor = my_node.neighbors[idx];

          if ( my_node.dist + 1 < neighbor.dist ) {
            neighbor.dist = my_node.dist + 1;
            xpc::spawn( neighbor.idx, func );
          }
        }});}
  \end{Verbatim}
  \end{subfigure}

  \caption{\BF{Example Code Using XPC Programming API --} Preliminary
    ideas for how the XPC programming API could be used to parallelize
    amorphous data parallel applications. The \TT{spawn} primitive is
    used to parallelize the breadth-first search application using a
    divide-and-conquer algorithm.}
  \label{fig-xpc-api}

%\end{figure*}
