%=========================================================================
% sec-conclusion
%=========================================================================

\section{Conclusion}
\label{sec-conclusion}

The XPC architecture provides a fundamentally different abstraction for
heterogeneous architectures by exposing fine-grain parallel tasks as
parallel function calls in the software/hardware interface. By exploring
both software and hardware approaches for exposing, scheduling, and
executing fine-grain parallel tasks, it is possible to identify
permutations yielding optimal performance and energy efficiency for a
wide range of amorphous data parallel applications with reasonable
hardware complexity overhead.  The long-term plan is to first develop the
XPC software framework to profile benchmarks on an x86-based platform,
then design and evaluate each XPC tile in isolation, before combining
both software and hardware elements together to explore the adaptive
scheduling of fine-grain tasks across the heterogeneous substrate. A
subset of the XPC architecture will also be implemented at the
register-transfer level to better estimate area, energy, and timing
metrics.
